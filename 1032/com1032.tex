\documentclass{article}
\author{Jim S. Lam}
\usepackage{hyperref}
\usepackage{amsmath}
\usepackage{amssymb}
\usepackage{tikz}
\usetikzlibrary{automata, positioning, arrows}

\usepackage{listings}
\usepackage{color}
\usepackage[edges]{forest}
\usetikzlibrary{arrows.meta}



\definecolor{dkgreen}{rgb}{0,0.6,0}
\definecolor{gray}{rgb}{0.5,0.5,0.5}
\definecolor{mauve}{rgb}{0.58,0,0.82}

\lstset{frame=tb,
  language=Java,
  aboveskip=3mm,
  belowskip=3mm,
  showstringspaces=false,
  columns=flexible,
  basicstyle={\small\ttfamily},
  numbers=none,
  numberstyle=\tiny\color{gray},
  keywordstyle=\color{blue},
  commentstyle=\color{dkgreen},
  stringstyle=\color{mauve},
  breaklines=true,
  breakatwhitespace=true,
  tabsize=3
}

\begin{document}
\title{COM1032 OPERATING SYSTEMS}
\maketitle
\tableofcontents
\pagebreak
\section{OPERATING SYSTEM DEFINITION}

An Operating System is a program that controls the execution of user programs and acts as an intermediary between users and computer hardware.

It is a software layer between application programs and computer hardware.

\begin{forest}
    for tree={
      align=center,
      font=\sffamily,
      edge+={thick, -{Stealth[]}},
      l sep'+=10pt,
      fork sep'=10pt,
    },
    forked edges,
    if level=0{
      inner xsep=0pt,
      tikz={\draw [thick] (.children first) -- (.children last);}
    }{},
    [Hardware
      [
        System software (Operating systems)
        [
            Application software
            [
                Users
            ]
        ]
      ]
    ]
  \end{forest}

\end{document}