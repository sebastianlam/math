\documentclass{article}

\usepackage{amsmath}

\usepackage{amssymb} 

\begin{document}

\section{Set Notation}

Sets are denoted with capital letters, e.g. A, B, C. The elements of a set are listed inside curly brackets:

\begin{align*}  
A &= \{1, 2, 3\} \\
B &= \{a, b, c\}  
\end{align*}

The union of two sets A and B is denoted $A \\cup B$ and contains all elements of both sets. The intersection $A \\cap B$ contains elements common to both.

\begin{align*}
A \cup B &= \{1, 2, 3, a, b, c\} \\ 
A \cap B &= \emptyset
\end{align*}

Sets can also be described using set builder notation:

\begin{align*}
C &= \{x \mid x \in \mathbb{N}, 0 \leq x \leq 5\} \\
&= \{x \mid \textrm{$x$ is natural}, 0 \leq x \leq 5\}  
\end{align*}

This covers the basics of set notation and operations in LaTeX math mode. Additional set theory topics like power sets, Cartesian products, etc. could be added.

\begin{align*}
\textrm{bunnies} &= E(n\_{g+1}' \mid n\_i''; 1\le i\le g) \\
               &= \{\textrm{Who knows, it's all pipes!}\}  
\end{align*}

COM1031 WEEK 2

\begin{align*} 
\{a, b\} &= \{b, a\} \qquad \textrm{ Still talking about sets.}
\end{align*}

About the size of Cartesian products: 

\begin{align*}
\textrm{ayy} \\
\lvert \mathbf{A} \rvert \cdot \lvert \mathbf{B} \rvert &= \lvert \mathbf{A} \times \mathbf{B} \rvert \\
\mathbf{R}^{-1} &= \{(b, a) \in \mathbf{B} \times \mathbf{A} \mid (a, b) \in \mathbf{R}\}
\end{align*}

Let S be the set of students.\\
Let B be the set of bars in Guildford.\\  
V(x,y) means x has visited y

\begin{align*}  
\exists x \in \mathbf{S}. \forall y \in \mathbf{B}.V(x,y)
\end{align*}

Let $\mathbf{S}$ be the set of students. \\
Let $\mathbf{B}$ be the set of bars in Guildford. \\
T(x) means x studies hard. \\
D(x) means x does well in exams.

\begin{align*}
(\forall x \in \mathbf{S}.T(x)) \implies (\forall s \in \mathbf{S}.D(s))  
\end{align*}

\begin{align*}
\exists a \in \mathbf{A}.D(a).G(a) \land \exists b \in \mathbf{A}.D(b).G(b)
\end{align*}

\end{document}