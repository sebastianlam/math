\documentclass{article}
\usepackage{amsmath}
\usepackage{amssymb}
\begin{document}
\section{Set Theory}
\subsection{Introduction}
Quick recap on Naive set theory, meaning:

1. Introducing the basic concept of sets;

2. Introduce notation;

3. Illustrate Union, Intersection, and Set Difference operations;

4. Venn diagrams as proof;

5. Power sets;

6. How to proof with more rigor.

Question: Why are sets relevant to computing?

We have to represent data to compute it.

To group data, we put it into sets.

Some sets that I have seen before:
\begin{align*}
    \text{The set of natural numbers: } \mathbb{N} & = \{1, 2, 3, ...\}                    \\
    \text{The set of integers: } \mathbb{Z}        & = \{..., -3, -2, -1, 0, 1, 2, 3,...\}
\end{align*}
Question: What are sets?

Sets are a collection of unique elements.

Sets are denoted with capital letters, e.g. A, B, C. The elements of a set are listed inside curly brackets:
\begin{align*}
    A & = \{1, 2, 3\}                \\
    B & = \{a, b, c, d, e, f, g, h\}
\end{align*}
The union of two sets A and B is denoted $A \cup B$ and contains all elements of both sets. The intersection $A \cap B$ contains elements common to both.
\begin{align*}
    A \cup B           & = \{1, 2, 3, a, b, c\}   \\
    A \bigtriangleup B & = \emptyset \bigcup \{\}
\end{align*}
Sets can also be described using set builder notation:
\begin{align*}
    C & = \{x | x \in \mathbb{N}, 0 \leq x \leq 5\}                    \\
      & = \{x |\text{$x$ is in the appropriate set}, 0 \leq x \leq 5\}
\end{align*}
This covers the basics of set notation and operations in Latex math mode. Additional set theory topics like power sets, Cartesian products, etc. could be added.
\begin{align*}
     & = \{\text{Who knows, it's all pipes!}\}
\end{align*}
\end{document}
