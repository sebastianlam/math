\documentclass{article}
\usepackage{amsmath}
\usepackage{amssymb}
\begin{document}
\section{Set Theory}
\subsection{Introduction}
Quick recap on Naive set theory, meaning; \\
1. Introducting the basic concept of sets; \\
2. Introduce notation; \\
3. Illustrate Union, Intersection, and Set Diffrenece operations; \\
4. Venn diagrams as proof; \\
5. Power sets; \\
6. How to proof with more rigor. \\\\
Question: Why are sets relevant to computing? \\
We have to represent data to compute it. \\
To group data, we put it into sets. \\\\
Some sets that I have seen before:
\begin{align*}
\textrm{The set of natural numbers}:\mathbb{N} &= \{1, 2, 3, ...\}\\
\textrm{The set of integers}:\mathbb{Z} &= \{..., -3, -2, -1, 0, 1, 2, 3,...\} \\
\end{align*}
Sets are denoted with capital letters,
e.g. A, B, C. The elements of a set are listed inside curly brackets:
\begin{align*}
A &= \{1, 2, 3\} \\
B &= \{a, b, c, d, e, f, g, h\} \\
&= \int_{a}^{b} x^2 \,dx\ in = he
\end{align*}
The union of two sets A and B is denoted $A \cup B$ and contains all elements of both sets. The intersection $A \cap B$ contains elements common to both.
\begin{align*}
A \cup B & = \{1, 2, 3, a, b, c\} \\
A \bigtriangleup B & = \emptyset \bigcup \{\}
\end{align*}
Sets can also be described using set builder notation:
\begin{align*}
C & = \{x | x \in \mathbb{N}, 0 \leq x \leq 5\} \\
& = \{x |\textrm{$x$ is in asdfiuh is asfe}, 0 \leq x \leq 5\}
\end{align*}
This covers the basics of set notation and operations in Latex math mode. Additional set theory topics like power sets, Cartesian products, etc. could be added.
\begin{align*}
bunnies & = E(n_{g+1}’|n_i'';\,1\le i\le g) \\
& = \{\textrm{Who knows, it's all pipes!}\}
\end{align*}
\end{document}