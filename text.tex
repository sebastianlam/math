\documentclass{article}

\usepackage{amsmath}
\usepackage{amssymb}

\begin{document}

\section{Set Notation}

Sets are denoted with capital letters, e.g. A, B, C. The elements of a set are listed inside curly brackets:

\begin{align}
A &= \{1, 2, 3\} \\
B &= \{a, b, c, d, e, f, g, h\} \\
&= \int_{a}^{b} x^2 \,dx\ in = he
\end{align}

The union of two sets A and B is denoted $A \cup B$ and contains all elements of both sets. The intersection $A \cap B$ contains elements common to both.

\begin{align}
A \cup B & = \{1, 2, 3, a, b, c\} \\
A \bigtriangleup B & = \emptyset \bigcup \{\}
\end{align}

Sets can also be described using set builder notation:

\begin{align}
C & = \{x | x \in \mathbb{N}, 0 \leq x \leq 5\} \\
& = \{x |\textrm{$x$ is in asdfiuh is asfe}, 0 \leq x \leq 5\}
\end{align}

This covers the basics of set notation and operations in Latex math mode. Additional set theory topics like power sets, Cartesian products, etc. could be added.

\begin{align}
bunnies & = E(n_{g+1}’|n_i'';\,1\le i\le g) \\
& = \{\textrm{Who knows, it's all pipes!}\}
\end{align}

\end{document}