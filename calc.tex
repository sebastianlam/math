\documentclass[12pt,a4paper]{article}
\usepackage[utf8]{inputenc}
\usepackage[T1]{fontenc}
\usepackage{amsmath,amssymb,amsfonts,bm,mathtools,physics}
\usepackage{graphicx}
\usepackage{hyperref}
\usepackage{xcolor}
\usepackage{tcolorbox}
\usepackage{titlesec}

\colorlet{sectioncolor}{blue!70!black}
\colorlet{subsectioncolor}{blue!50!black}

\titleformat{\section}
{\color{sectioncolor}\normalfont\Large\bfseries}
{\color{sectioncolor}\thesection}{1em}{}

\titleformat{\subsection}
{\color{subsectioncolor}\normalfont\large\bfseries}
{\color{subsectioncolor}\thesubsection}{1em}{}

\newcommand{\eqnote}[1]{\textcolor{gray}{\textrm{#1}}}

\title{\textcolor{blue!70!black}{\Huge\textbf{Advanced Calculus Notes}}}
\author{\Large Your Name}
\date{}

\begin{document}

\maketitle

\section{Introduction}
Welcome to these comprehensive calculus notes. This document is created using \LaTeX, showcasing its power in typesetting mathematical content.

\section{Fundamental Concepts}

\begin{tcolorbox}[colback=blue!5!white,colframe=blue!75!black,title=Key Formulas]
\begin{flalign*}
    & \textbf{Differentiation from first principles:} &&&\\
    & \quad f'(x) = \lim_{h \to 0} \frac{f(x + h) - f(x)}{h} &&&\\
    & \eqnote{(The foundation of differential calculus)} &&&\\[1em]
    & \dv{x}\sin(x) = \cos(x) \quad \eqnote{(Sine derivative)} &&&\\
    & \dv{x}\cos(x) = -\sin(x) \quad \eqnote{(Cosine derivative)} &&&\\
    & \dv{x}e^x = e^x \quad \eqnote{(Exponential function derivative)} &&&\\
    & \dv{x}\ln x = \frac{1}{x} \quad \eqnote{(Natural logarithm derivative)} &&&\\
    & \dv{x}\tan(kx) = k\sec^2(kx) \quad \eqnote{(Tangent derivative)} &&&\\
    & \dv{x}\sec(kx) = k\sec(kx)\tan(kx) \quad \eqnote{(Secant derivative)} &&&\\
    & \dv{x}\cot(kx) = -k\csc^2(kx) \quad \eqnote{(Cotangent derivative)} &&&\\
    & \dv{x}\csc(kx) = -k\csc(kx)\cot(kx) \quad \eqnote{(Cosecant derivative)} &&&\\
\end{flalign*}
\end{tcolorbox}

\section{Advanced Differentiation Rules}

\begin{tcolorbox}[colback=green!5!white,colframe=green!75!black,title=Chain Rule and Product Rule]
\begin{flalign*}
    & \textbf{Chain Rule:} &&&\\
    & \quad \dv{x}f(g(x)) = \dv{g}{x} \cdot \dv{f}{g} &&&\\
    & \eqnote{(Differentiating composite functions)} &&&\\[1em]
    & \textbf{Product Rule:} &&&\\
    & \quad \dv{x}[f(x) \cdot g(x)] = f(x) \dv{g}{x} + g(x) \dv{f}{x} &&&\\
    & \eqnote{(Differentiating the product of two functions)} &&&\\[1em]
    & \textbf{Quotient Rule:} &&&\\
    & \quad \dv{x}\frac{f(x)}{g(x)} = \frac{g(x) \dv{f}{x} - f(x) \dv{g}{x}}{[g(x)]^2} &&&\\
    & \eqnote{(Differentiating the quotient of two functions)} &&&
\end{flalign*}
\end{tcolorbox}

\section{Limits}
Limits form the foundation of calculus, describing the behavior of functions as they approach certain values.

\section{Integrals}
Integration is the reverse process of differentiation, used to find areas, volumes, and solutions to differential equations.

\section{Multivariable Calculus}
Extending calculus concepts to functions of multiple variables.

\subsection{Partial Derivatives}
Derivatives with respect to one variable while holding others constant.

\subsection{Gradient}
The vector of partial derivatives, representing the direction of steepest ascent.

\subsection{Divergence and Curl}
Measures of a vector field's expansion and rotation.

\subsection{Multiple Integrals}
Integrating over regions in multiple dimensions.

\subsection{Vector Calculus Theorems}

\begin{tcolorbox}[colback=red!5!white,colframe=red!75!black,title=Important Theorems]
\begin{itemize}
    \item \textbf{Green's Theorem:} Relates line integrals to double integrals.
    \item \textbf{Stokes' Theorem:} Generalizes Green's Theorem to 3D.
    \item \textbf{Divergence Theorem:} Relates surface integrals to triple integrals.
\end{itemize}
\end{tcolorbox}

\end{document}