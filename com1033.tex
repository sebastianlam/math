\documentclass{article}
\author{Jim S. Lam}
\usepackage{hyperref}
\usepackage{amsmath}
\usepackage{amssymb}
\usepackage{tikz}
\usetikzlibrary{automata, positioning, arrows}

\usepackage{listings}
\usepackage{color}

\definecolor{dkgreen}{rgb}{0,0.6,0}
\definecolor{gray}{rgb}{0.5,0.5,0.5}
\definecolor{mauve}{rgb}{0.58,0,0.82}

\lstset{frame=tb,
  language=Java,
  aboveskip=3mm,
  belowskip=3mm,
  showstringspaces=false,
  columns=flexible,
  basicstyle={\small\ttfamily},
  numbers=none,
  numberstyle=\tiny\color{gray},
  keywordstyle=\color{blue},
  commentstyle=\color{dkgreen},
  stringstyle=\color{mauve},
  breaklines=true,
  breakatwhitespace=true,
  tabsize=3
}

\begin{document}
\title{COM1033 FOUNDATIONS OF COMPUTING II}
\maketitle
\tableofcontents
\pagebreak
\section{Vectors}
\subsection{Vector Definition}

Let n $\in$ $\mathbb{N}$ and $n$ $>$ 0.

The set of all vectors is the cartesian product of $\mathbb{R}$ by $n$ times, which is a set of ordered $n$-tuples of real numbers.

\begin{align*}
    \mathbb{R}^3 = \{ (x, y, z) \mid x, y, z \in \mathbb{R} \}
\end{align*}

\subsection{Vector Operations}
\subsubsection{Addition}
\begin{align*}
    \begin{pmatrix} a \\ b \\ c \end{pmatrix} + \begin{pmatrix} d \\ e \\ f \end{pmatrix} = \begin{pmatrix} a + d \\ b + e \\ c + f \end{pmatrix}
\end{align*}
\subsubsection{Scalar Multiplication}
\begin{align*}
    \lambda
    \begin{pmatrix}
        a \\ b \\ c
    \end{pmatrix}
    =
    \begin{pmatrix}
        \lambda a \\ \lambda b \\ \lambda c
    \end{pmatrix}
\end{align*}
\subsubsection{Dot Product / Scalar Product}
\begin{align*}
    \begin{pmatrix} a \\ b \\ c \end{pmatrix} \cdot \begin{pmatrix} d \\ e \\ f \end{pmatrix} = (a \cdot d) + (b \cdot e) + (c \cdot f)
\end{align*}
\subsubsection{Linear Combination}
Let $ \lambda_1, \lambda_2, ... , \lambda_n $ be $n$ scalars, and $ \vec{v_1}, \vec{v_2}, ..., \vec{v_n} $ be $n$ vectors.
\begin{align*}
    \vec{w} = \lambda_1 \vec{v_1} + \lambda_2 \vec{v_2} + ... + \lambda_n \vec{v_n}
\end{align*}
$ \vec{w} $ is a linear combination of $ \vec{v_1}, \vec{v_2}, ..., \vec{v_n} $ using the scalars $ \lambda_1, \lambda_2, ... , \lambda_n $.

\subsubsection{Linear Dependence}

Let there be $n$ vectors of the same dimension.

If the null vector $ \vec{o} $ can be expressed as linear combination of the $n$ vectors as defined, using non null scalars.

In other words, the $n$ vectors are linearly dependent if:

\begin{align*}
    \vec{w} = \lambda_1 \vec{v_1} + \lambda_2 \vec{v_2} + ... + \lambda_n \vec{v_n} \mid \exists \lambda_1, \lambda_2, ..., \lambda_n \neq 0, 0, ..., 0
\end{align*}
\subsubsection{Matricies}

Matrices are defined as a table where it's elements have two indicies, limited to the size of the matrix size.



\subsubsection{Exercises}
Question 1:
Sum the following vectors $\in \mathbb{R}^3$:
\begin{align*}
    % \textrm{I forgot.}
    \vec{v_1} &= \begin{pmatrix} 3 \\ 5 \\ -4 \end{pmatrix}, \vec{v_2} = \begin{pmatrix} 0 \\ 1 \\ 4 \end{pmatrix} \\
    \textrm{Calculate the product } \lambda \vec{v_1} \textrm{ with } \lambda &= 2 \\
    \lambda \vec{v_1} &= \begin{pmatrix} 6 \\ 10 \\ -8 \end{pmatrix}
\end{align*}
Question 2
\begin{align*}
    \vec{u} &= \begin{pmatrix} 3 \\ 5 \\ -4 \end{pmatrix} \quad \vec{v} = \begin{pmatrix} 2 \\ 2 \\ 4 \end{pmatrix} \\
    \vec{u} \dot \vec{v} &= 3 \cdot 2 + 5 \cdot 2 + -4 \cdot 4 \\
    &= 6 + 10 - 16 \\
    &= 0
\end{align*}
Question 3
\begin{align}
    \vec{v_1} = \begin{pmatrix} 1 \\ 2 \\ 1 \end{pmatrix} \quad \vec{v_2} = \begin{pmatrix} 0 \\ 2 \\ 2 \end{pmatrix} \quad \vec{v_3} = \begin{pmatrix} 1 \\ 6 \\ 5 \end{pmatrix} \\
    \textrm{when: } \lambda_1 = 1, \lambda_2 = 2, \lambda_3 = -1 \\
    \lambda_1 \vec{v_1} + \lambda_2 \vec{v_2} + \lambda_3 \vec{v_3} &= \begin{pmatrix} 0 \\ 0 \\ 0 \end{pmatrix}
\end{align}
Question 4:

Let $v_1, v_2, ..., v_n$ be n linearly independent vectors.
Consider the the set of scalers $\lambda_1, \lambda_2, ..., \lambda_n$ such that $\lambda_1 v_1 + \lambda_2 v_2 + ... + \lambda_n v_n = 0$.
Find alternative sets of the scalers.
\begin{align*}
    \textrm{Just multiply all the scalars by a common scaling factor, let's say } \mu
\end{align*}

Question 5:
\begin{align*}
    \textrm{} \\
    a_3 \textrm{ is on the same line}
\end{align*}

Question 6:

\begin{align}
(1, 0, 2 \\
3,5,1 \\
2,2,0) \\
\end{align}

\end{document}