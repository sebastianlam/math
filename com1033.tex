\documentclass{article}
\author{Jim S. Lam}
\usepackage{hyperref}
\usepackage{amsmath}
\usepackage{amssymb}
\usepackage{tikz}
\usetikzlibrary{automata, positioning, arrows}

\usepackage{listings}
\usepackage{color}

\definecolor{dkgreen}{rgb}{0,0.6,0}
\definecolor{gray}{rgb}{0.5,0.5,0.5}
\definecolor{mauve}{rgb}{0.58,0,0.82}

\lstset{frame=tb,
  language=Java,
  aboveskip=3mm,
  belowskip=3mm,
  showstringspaces=false,
  columns=flexible,
  basicstyle={\small\ttfamily},
  numbers=none,
  numberstyle=\tiny\color{gray},
  keywordstyle=\color{blue},
  commentstyle=\color{dkgreen},
  stringstyle=\color{mauve},
  breaklines=true,
  breakatwhitespace=true,
  tabsize=3
}

\begin{document}
\title{COM1033 FOUNDATIONS OF COMPUTING II}
\maketitle
\tableofcontents
\pagebreak
\section{Vectors}
\subsection{Vector Definition}

Let n $\in$ $\mathbb{N}$ and $n$ $>$ 0.

The set of all vectors is the cartesian product of $\mathbb{R}$ by $n$ times, which is a set of ordered $n$-tuples of real numbers.

\begin{align*}
    \mathbb{R}^3 = \{ (x, y, z) \mid x, y, z \in \mathbb{R} \}
\end{align*}

\subsection{Vector Operations}
\subsubsection{Addition}
\begin{align*}
    \begin{pmatrix} a \\ b \\ c \end{pmatrix} + \begin{pmatrix} d \\ e \\ f \end{pmatrix} = \begin{pmatrix} a + d \\ b + e \\ c + f \end{pmatrix}
\end{align*}
\subsubsection{Scalar Multiplication}
\begin{align*}
    \lambda
    \begin{pmatrix}
        a \\ b \\ c
    \end{pmatrix}
    =
    \begin{pmatrix}
        \lambda a \\ \lambda b \\ \lambda c
    \end{pmatrix}
\end{align*}
\subsubsection{Dot Product / Scalar Product}
\begin{align*}
    \begin{pmatrix} a \\ b \\ c \end{pmatrix} \cdot \begin{pmatrix} d \\ e \\ f \end{pmatrix} = (a \cdot d) + (b \cdot e) + (c \cdot f)
\end{align*}
\subsubsection{Linear Combination}
Let $ \lambda_1, \lambda_2, ... , \lambda_n $ be $n$ scalars, and $ \vec{v_1}, \vec{v_2}, ..., \vec{v_n} $ be $n$ vectors.
\begin{align*}
    \vec{w} = \lambda_1 \vec{v_1} + \lambda_2 \vec{v_2} + ... + \lambda_n \vec{v_n}
\end{align*}
$ \vec{w} $ is a linear combination of $ \vec{v_1}, \vec{v_2}, ..., \vec{v_n} $ using the scalars $ \lambda_1, \lambda_2, ... , \lambda_n $.

\subsubsection{Linear Dependence}

Let there be $n$ vectors of the same dimension.

If the null vector $ \vec{o} $ can be expressed as linear combination of the $n$ vectors as defined, using non null scalars.

In other words, the $n$ vectors are linearly dependent if:

\begin{align*}
    \vec{w} = \lambda_1 \vec{v_1} + \lambda_2 \vec{v_2} + ... + \lambda_n \vec{v_n} \mid \exists \lambda_1, \lambda_2, ..., \lambda_n \neq 0, 0, ..., 0
\end{align*}
\subsubsection{Matricies}

Matrices are defined as a table where it's elements have two indicies, limited to the size of the matrix size.



\subsubsection{Exercises}
Question 1:
Sum the following vectors $\in \mathbb{R}^3$:
\begin{align*}
    % \textrm{I forgot.}
    \vec{v_1}                                                                 & = \begin{pmatrix} 3 \\ 5 \\ -4 \end{pmatrix}, \vec{v_2} = \begin{pmatrix} 0 \\ 1 \\ 4 \end{pmatrix} \\
    \textrm{Calculate the product } \lambda \vec{v_1} \textrm{ with } \lambda & = 2                                                                                                 \\
    \lambda \vec{v_1}                                                         & = \begin{pmatrix} 6 \\ 10 \\ -8 \end{pmatrix}
\end{align*}
Question 2
\begin{align*}
    \vec{u}              & = \begin{pmatrix} 3 \\ 5 \\ -4 \end{pmatrix} \quad \vec{v} = \begin{pmatrix} 2 \\ 2 \\ 4 \end{pmatrix} \\
    \vec{u} \dot \vec{v} & = 3 \cdot 2 + 5 \cdot 2 + -4 \cdot 4                                                                   \\
                         & = 6 + 10 - 16                                                                                          \\
                         & = 0
\end{align*}
Question 3
\begin{align}
    \vec{v_1} = \begin{pmatrix} 1 \\ 2 \\ 1 \end{pmatrix} \quad \vec{v_2} = \begin{pmatrix} 0 \\ 2 \\ 2 \end{pmatrix} \quad \vec{v_3} = \begin{pmatrix} 1 \\ 6 \\ 5 \end{pmatrix} \\
    \textrm{when: } \lambda_1 = 1, \lambda_2 = 2, \lambda_3 = -1                                                                                                                  \\
    \lambda_1 \vec{v_1} + \lambda_2 \vec{v_2} + \lambda_3 \vec{v_3} & = \begin{pmatrix} 0 \\ 0 \\ 0 \end{pmatrix}
\end{align}
Question 4:

Let $v_1, v_2, ..., v_n$ be n linearly independent vectors.
Consider the the set of scalers $\lambda_1, \lambda_2, ..., \lambda_n$ such that $\lambda_1 v_1 + \lambda_2 v_2 + ... + \lambda_n v_n = 0$.
Find alternative sets of the scalers.
\begin{align*}
    \textrm{Just multiply all the scalars by a common scaling factor, let's say } \mu
\end{align*}

Question 5:
\begin{align*}
    \textrm{} \\
    a_3 \textrm{ is on the same line}
\end{align*}

\section{Matrices}


\subsection{Matrix Multiplication}

Let $A$ be a $m \times n$ matrix, and $B$ be a $n \times p$ matrix.


Question 6:

\begin{align}
    \begin{pmatrix}1, 0, 2 \\
        3,5,1   \\
        2,2,0   \\
    \end{pmatrix}            \\
    0 - 2 - 0 + 0 + 12 - 20 = -10     \\
    \begin{pmatrix}
        1 , 0 , 3 \\
        1, -1, 0  \\
        4, 2, 1   \\
    \end{pmatrix}                    \\
    -1 -(0) -(0) + 0  + 6 -(-12) = 17 \\
    v_1 + 2v_2 - v_3 \textrm{ i.e.:}  \\
    \begin{pmatrix} 1 \\ 2 \\ 1 \\ \end{pmatrix} + 2 \begin{pmatrix} 0 \\ 1 \\ 2 \\ \end{pmatrix} - \begin{pmatrix} 3 \\ 0 \\ 1 \\ \end{pmatrix} = \begin{pmatrix} -1 \\ 3 \\ -1 \\ \end{pmatrix}
    -1 -(0) -(0) + 0 + 6 -(-12) = 17  \\
\end{align}

Start with a matrix with a determinant of 0.

\begin{align}
    \begin{pmatrix}
        a_11 & a_12 & a_13 \\
        a_21 & a_22 & a_23 \\
        a_31 & a_32 & a_33 \\
    \end{pmatrix} \\
    \mathbb{M} =
    \begin{pmatrix}
        1 & 0 & 2 \\
        3 & 5 & 1 \\
        2 & 2 & 0 \\
    \end{pmatrix}
\end{align}

the first laplace theorem is to expand the determinant of a matrix along the first row.

\begin{align}
    a_11 \begin{pmatrix}
             5 & 1 \\
             2 & 0 \\
         \end{pmatrix}
    - a_12 \begin{pmatrix}
               3 & 1 \\
               2 & 0 \\
           \end{pmatrix}
    + a_13 \begin{pmatrix}
               3 & 5 \\
               2 & 2 \\
           \end{pmatrix}
\end{align}

for the example matrix $\mathbb{M}$:

\begin{align}
    \det(\mathbb{M}) = 6 - 6 - 0 + 6 + 0 - 12 = -6 \\
\end{align}

2. Yes

question 4: 2

Row three is a null row. \\
has determinant 0 \\
Row three plus a row one multiplied by the some scalar has the same determinant. \\
Row three plus row two multiplied by the some scalar has the same determinant. \\
-170 \textrm{for both det(}AB\textrm{) and det(}BA\textrm{)}

\section{Linear Equations}

\subsection{System of Linear Equations}

\begin{align}
    \begin{cases}
        a_{1,1}x + a_{1,2}y = b_1 \\
        a_{2,1}x + a_{2,2}y = b_2
    \end{cases}
\end{align}

the above can be expressed as a matrix equation:

\begin{align}
    \begin{pmatrix}
        a_{1,1} & a_{1,2} \\
        a_{2,1} & a_{2,2}
    \end{pmatrix}
    \begin{pmatrix}
        x \\
        y
    \end{pmatrix}
    =
    \begin{pmatrix}
        b_1 \\
        b_2
    \end{pmatrix}
\end{align}

Let's use a 4 by 4 square matrix as an example:

\begin{align}
    \begin{pmatrix}
        a_{1,1} & a_{1,2} & a_{1,3} & a_{1,4} \\
        a_{2,1} & a_{2,2} & a_{2,3} & a_{2,4} \\
        a_{3,1} & a_{3,2} & a_{3,3} & a_{3,4} \\
        a_{4,1} & a_{4,2} & a_{4,3} & a_{4,4} \\
    \end{pmatrix}
    \begin{pmatrix}
        x \\
        y \\
        z \\
        w \\
    \end{pmatrix}
    =
    \begin{pmatrix}
        b_1 \\
        b_2 \\
        b_3 \\
        b_4 \\
    \end{pmatrix}
\end{align}

/subsubsection{Kramer's method}

For the above 4 by 4 matrix, we can solve for $x$ by using the following formula:

\begin{align}
    x = \frac{\det(\mathbb{M}_x)}{\det(\mathbb{M})}
\end{align}

note that $\mathbb{M}_x$ is the matrix $\mathbb{M}$ with the first column replaced by the column vector $\mathbb{B}$.

So $ x_1 $ for instance would be:

\begin{align}
    x_1 = \frac{\det(\mathbb{M}_1)}{\det(\mathbb{M})} \\
    \textrm{Where } \mathbb{M}_1 =
    \begin{pmatrix}
        b_1 & a_{1,2} & a_{1,3} & a_{1,4} \\
        b_2 & a_{2,2} & a_{2,3} & a_{2,4} \\
        b_3 & a_{3,2} & a_{3,3} & a_{3,4} \\
        b_4 & a_{4,2} & a_{4,3} & a_{4,4} \\
    \end{pmatrix}
\end{align}

\subsubsection{Mock test}

\begin{align*}
    \begin{cases}
        3x - 2y + z = 2 \\
        2z = 2          \\
        x + y = 2       \\
    \end{cases}
    \textrm{Can be expressed as:}              \\
    \begin{pmatrix}
        3 & -2 & 1 \\
        0 & 0  & 2 \\
        1 & 1  & 0 \\
    \end{pmatrix}
    \begin{pmatrix}
        x \\
        y \\
        z \\
    \end{pmatrix}
    =
    \begin{pmatrix}
        2 \\
        2 \\
        2 \\
    \end{pmatrix}
    \textrm{the determinant of the matrix is:} \\
    0 - 6 - 0 - 4 + 0 - 0 = -10                \\
    \textrm{It is solvable.}                   \\
    \textrm{The solution for x is:}            \\
    \det(\mathbb{M}_x) =
    \det(\begin{pmatrix}
                 2 & -2 & 1 \\
                 2 & 0  & 2 \\
                 2 & 1  & 0 \\
             \end{pmatrix})
    = 0 - 4 + 0 - 8 + 2 - 0 = -10              \\
    = \frac{-10}{-10} = 1                      \\
    \textrm{Question 2.}                       \\
    \begin{pmatrix}
        3 & -2 & 1 \\
        2 & 0  & 2 \\
        1 & 1  & 2 \\
    \end{pmatrix}
    \begin{pmatrix}
        x \\
        y \\
        z \\
    \end{pmatrix}
    =
    \begin{pmatrix}
        2 \\
        2 \\
        2 \\
    \end{pmatrix}
    \textrm{the determinant of the matrix is:} \\
    0 - 6 + 8 - 4 + 2 + 0 = 0                  \\
    \textrm{It is not solvable.}               \\
    \textrm{Question 3.}                       \\
    \det(
    \begin{pmatrix}
            1 & 1  & 0  & 1 \\
            4 & 1  & -1 & 0 \\
            2 & -1 & 1  & 2 \\
        \end{pmatrix})
    =
    \det(
    \begin{pmatrix}
            1 & 1  & 0  & 1 \\
            4 & 1  & -1 & 0 \\
            0 & -3 & 1  & 0 \\
        \end{pmatrix})
    =
    \det(
    \begin{pmatrix}
            1 & 1  & 0  & 1  \\
            0 & -3 & -1 & -4 \\
            0 & -3 & 1  & 0  \\
        \end{pmatrix}
    )
    =
    \det(
    \begin{pmatrix}
            1 & 1  & 0  & 1  \\
            0 & -3 & -1 & -4 \\
            0 & 0  & 2  & 4  \\
        \end{pmatrix}
    )                                          \\
    z = 2                                      \\
    -3y -1(2) = -4                             \\
    -3y = -2                                   \\
    y = \frac{2}{3}                            \\
    x = \frac{1}{3}                            \\
    \textrm{Question 4.}                       \\
    \begin{pmatrix}
        0 & a & b \\
        c & 0 & d \\
        e & f & 0 \\
    \end{pmatrix}
\end{align*}

\section{Vector Spaces}

Fuck you

Let $\mathbf{E}$ be a non-null set $ (\mathbf{E} \neq \emptyset) $ and $ \mathbb{K}$ be a scalar set.

We designate vectors as elements of $\mathbf{E}$

let '+' be an internal compositional law, i.e. $ \mathbf{E} \times \mathbf{E} \rightarrow \mathbf{E}$

let '.' be an internal compositional law, i.e. $ \mathbb{K} \times \mathbf{E} \rightarrow \mathbf{E}$


\end{document}