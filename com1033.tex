\documentclass{article}
\author{Jim S. Lam}
\usepackage{hyperref}
\usepackage{amsmath}
\usepackage{amssymb}
\usepackage{tikz}
\usetikzlibrary{automata, positioning, arrows}

\usepackage{listings}
\usepackage{color}

\definecolor{dkgreen}{rgb}{0,0.6,0}
\definecolor{gray}{rgb}{0.5,0.5,0.5}
\definecolor{mauve}{rgb}{0.58,0,0.82}

\lstset{frame=tb,
  language=Java,
  aboveskip=3mm,
  belowskip=3mm,
  showstringspaces=false,
  columns=flexible,
  basicstyle={\small\ttfamily},
  numbers=none,
  numberstyle=\tiny\color{gray},
  keywordstyle=\color{blue},
  commentstyle=\color{dkgreen},
  stringstyle=\color{mauve},
  breaklines=true,
  breakatwhitespace=true,
  tabsize=3
}

\begin{document}
\title{COM1033 FOUNDATIONS OF COMPUTING II}
\maketitle
\tableofcontents
\pagebreak
\section{Vectors}
\subsection{Vector Definition}

Let n $\in$ $\mathbb{N}$ and n $>$ 0.

The set of all vectors is the cartesian product of $\mathbb{R}$ by n times.

\begin{align*}
    \mathbb{R}^3 = \{ (x, y, z) \mid x, y, z \in \mathbb{R} \}
\end{align*}

\subsection{Vector Operations}
\subsubsection{Addition}
\subsubsection{Scalar Multiplication}
\subsubsection{Dot Product / Scalar Product}
\subsubsection{Exercises}
Question 1
\begin{align*}
    \textrm{I forgot.}
\end{align*}
Question 2
\begin{align*}
    \vec{u} &= \begin{pmatrix} 3 \\ 5 \\ -4 \end{pmatrix} \quad \vec{v} = \begin{pmatrix} 2 \\ 2 \\ 4 \end{pmatrix} \\
    \vec{u} \dot \vec{v} &= 3 \cdot 2 + 5 \cdot 2 + -4 \cdot 4 \\
    &= 6 + 10 - 16 \\
    &= 0
\end{align*}
Question 3
\begin{align}
    \vec{v_1} = \begin{pmatrix} 1 \\ 2 \\ 1 \end{pmatrix} \quad \vec{v_2} = \begin{pmatrix} 0 \\ 2 \\ 2 \end{pmatrix} \quad \vec{v_3} = \begin{pmatrix} 1 \\ 6 \\ 5 \end{pmatrix} \\
    \textrm{when: } \lambda_1 = 1, \lambda_2 = 2, \lambda_3 = -1 \\
    \lambda_1 \vec{v_1} + \lambda_2 \vec{v_2} + \lambda_3 \vec{v_3} &= \begin{pmatrix} 0 \\ 0 \\ 0 \end{pmatrix}
\end{align}
Question 4:

Let $v_1, v_2, ..., v_n$ be n linearly independent vectors.
Consider the the set of scalers $\lambda_1, \lambda_2, ..., \lambda_n$ such that $\lambda_1 v_1 + \lambda_2 v_2 + ... + \lambda_n v_n = 0$.
Find alternative sets of the scalers.
\begin{align*}
    \textrm{Just multiply all the scalars by a common scaling factor, let's say } \mu
\end{align*}

Question 5:
\begin{align*}
    \textrm{} \\
    a_3 \textrm{ is on the same line}
\end{align*}

\end{document}